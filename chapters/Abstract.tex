\title{Abstract}
\begin{abstract}
    Il rilevamento di anomalie su dati satellitari rappresenta una sfida fondamentale per la sostenibilità delle missioni spaziali ed il corretto funzionamento dei satelliti durante il loro periodo di azione.
    Per migliorare tali aspetti abbiamo implementato e analizzato due metodi basati su features extraction per timeseries, ROCKET e ROCKAD, portando un rilevamento efficiente di anomalie.

    I dataset presi in esame sono OPS\textunderscore SAT e NASA, per valutare le prestazioni dei modelli in termini di metriche, ovvero accuratezza, precisione, recupero, MCC, F1, AUC-ROC, AUC-PR e NScore; oltre a tenere in considerazione il tempo di esecuzione per valutarne anche l'efficienza, così da poterne consentire l'utilizzo anche a bordo dei satelliti, limitando il problema di un eccessivo scambio di informazioni tra essi e la base operativa sulla Terra.

    I risultati ottenuti formano una base da cui poter partire per esplorare nuovi contesti e confrontare future soluzioni, sempre più vicine a sistemi satellitari autonomi, riducendo drasticamente il bisogno di un intervento frequente e migliorando la sicurezza delle operazioni spaziali.
\end{abstract}
